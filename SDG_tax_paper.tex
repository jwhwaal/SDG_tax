%% 
%% Copyright 2007-2020 Elsevier Ltd
%% 
%% This file is part of the 'Elsarticle Bundle'.
%% ---------------------------------------------
%% 
%% It may be distributed under the conditions of the LaTeX Project Public
%% License, either version 1.2 of this license or (at your option) any
%% later version.  The latest version of this license is in
%%    http://www.latex-project.org/lppl.txt
%% and version 1.2 or later is part of all distributions of LaTeX
%% version 1999/12/01 or later.
%% 
%% The list of all files belonging to the 'Elsarticle Bundle' is
%% given in the file `manifest.txt'.
%% 
%% Template article for Elsevier's document class `elsarticle'
%% with harvard style bibliographic references

\documentclass[preprint,12pt,authoryear]{elsarticle}

%% Use the option review to obtain double line spacing
%% \documentclass[authoryear,preprint,review,12pt]{elsarticle}

%% Use the options 1p,twocolumn; 3p; 3p,twocolumn; 5p; or 5p,twocolumn
%% for a journal layout:
%% \documentclass[final,1p,times,authoryear]{elsarticle}
%% \documentclass[final,1p,times,twocolumn,authoryear]{elsarticle}
%% \documentclass[final,3p,times,authoryear]{elsarticle}
%% \documentclass[final,3p,times,twocolumn,authoryear]{elsarticle}
%% \documentclass[final,5p,times,authoryear]{elsarticle}
%% \documentclass[final,5p,times,twocolumn,authoryear]{elsarticle}

%% For including figures, graphicx.sty has been loaded in
%% elsarticle.cls. If you prefer to use the old commands
%% please give \usepackage{epsfig}

%% The amssymb package provides various useful mathematical symbols
\usepackage{amssymb}
%% The amsthm package provides extended theorem environments
%% \usepackage{amsthm}

%% The lineno packages adds line numbers. Start line numbering with
%% \begin{linenumbers}, end it with \end{linenumbers}. Or switch it on
%% for the whole article with \linenumbers.
%% \usepackage{lineno}

\journal{Journal of Cleaner Production}

\begin{document}

\begin{frontmatter}

%% Title, authors and addresses

%% use the tnoteref command within \title for footnotes;
%% use the tnotetext command for theassociated footnote;
%% use the fnref command within \author or \affiliation for footnotes;
%% use the fntext command for theassociated footnote;
%% use the corref command within \author for corresponding author footnotes;
%% use the cortext command for theassociated footnote;
%% use the ead command for the email address,
%% and the form \ead[url] for the home page:
%% \title{Title\tnoteref{label1}}
%% \tnotetext[label1]{}
%% \author{Name\corref{cor1}\fnref{label2}}
%% \ead{email address}
%% \ead[url]{home page}
%% \fntext[label2]{}
%% \cortext[cor1]{}
%% \affiliation{organization={},
%%            addressline={}, 
%%            city={},
%%            postcode={}, 
%%            state={},
%%            country={}}
%% \fntext[label3]{}

\title{Tax aggressiveness and corporate SDG reporting}

%% use optional labels to link authors explicitly to addresses:
\author[label1]{Johannes W.H. van der Waal, Thomas Thijssens}
\affiliation[label1]{organization={Open Universiteit of the Netherlands},
             addressline={Valkenburgerweg 177},
             city={Heerlen},
             postcode={6419 AT},
             state={},
             country={Netherlands}}

\begin{abstract}
We study the association between level of SDG reporting in CSR reports or annual reports and the level of tax aggressiveness. We argue that companies engaging with the SDGs for intrinsic, ethical reasons will be likely to shun tax aggressive behaviour and behave as responsible tax paying corproate citizens. In contrast, corporations reporting on the SDGs fo the purpose of reputation management, while in fact showing poor sustainable development performance would be more likely to show tax aggressive behaviour.

\end{abstract}

%%Graphical abstract
\begin{graphicalabstract}
%\includegraphics{grabs}
\end{graphicalabstract}

%%Research highlights
\begin{highlights}
\item Research highlight 1
\item Research highlight 2
\end{highlights}

\begin{keyword}
tax aggressiveness; SDGs; CSR reporting; sustainable development; corporate citizen

%% PACS codes here, in the form: \PACS code \sep code

%% MSC codes here, in the form: \MSC code \sep code
%% or \MSC[2008] code \sep code (2000 is the default)

\end{keyword}

\end{frontmatter}


%% main text
\section{Introduction}
\subsection{The role of the private sector in the 2030 Agenda}
%% describe the call on the private sector to contribute to the SDGs.
The private sector is acknowledged as a major driver of productivity, inclusive economic growth and job creation ()\cite{UnitedNationsGeneralAssembly2015b}). 

%%describe how corporoations communicate their CSR and SD policies and contributions

%%describe the reasons for CSR disclosure: signaling positive news to investors, seeking legitimacy if not performing well, controversy over CSR disclosure and performance

%%describe how corporations engage in SDG related disclosure

%%argue that the SDGs are an agenda setting framework that inspires business to become aware of their contributions to society sustainable development issues they affect or play a role in.
%%The 






\cite{Whait2018}


%% The Appendices part is started with the command \appendix;
%% appendix sections are then done as normal sections
%% \appendix



\bibliography{mybibfile}
\bibliographystyle{elsarticle-harv} 




\end{document}

\endinput
%%
%% End of file `elsarticle-template-harv.tex'.
